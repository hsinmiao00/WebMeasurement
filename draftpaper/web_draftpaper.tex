
%% bare_jrnl.tex
%% V1.4
%% 2012/12/27
%% by Michael Shell
%% see http://www.michaelshell.org/
%% for current contact information.
%%
%% This is a skeleton file demonstrating the use of IEEEtran.cls
%% (requires IEEEtran.cls version 1.8 or later) with an IEEE journal paper.
%%
%% Support sites:
%% http://www.michaelshell.org/tex/ieeetran/
%% http://www.ctan.org/tex-archive/macros/latex/contrib/IEEEtran/
%% and
%% http://www.ieee.org/



% *** Authors should verify (and, if needed, correct) their LaTeX system  ***
% *** with the testflow diagnostic prior to trusting their LaTeX platform ***
% *** with production work. IEEE's font choices can trigger bugs that do  ***
% *** not appear when using other class files.                            ***
% The testflow support page is at:
% http://www.michaelshell.org/tex/testflow/


%%*************************************************************************
%% Legal Notice:
%% This code is offered as-is without any warranty either expressed or
%% implied; without even the implied warranty of MERCHANTABILITY or
%% FITNESS FOR A PARTICULAR PURPOSE! 
%% User assumes all risk.
%% In no event shall IEEE or any contributor to this code be liable for
%% any damages or losses, including, but not limited to, incidental,
%% consequential, or any other damages, resulting from the use or misuse
%% of any information contained here.
%%
%% All comments are the opinions of their respective authors and are not
%% necessarily endorsed by the IEEE.
%%
%% This work is distributed under the LaTeX Project Public License (LPPL)
%% ( http://www.latex-project.org/ ) version 1.3, and may be freely used,
%% distributed and modified. A copy of the LPPL, version 1.3, is included
%% in the base LaTeX documentation of all distributions of LaTeX released
%% 2003/12/01 or later.
%% Retain all contribution notices and credits.
%% ** Modified files should be clearly indicated as such, including  **
%% ** renaming them and changing author support contact information. **
%%
%% File list of work: IEEEtran.cls, IEEEtran_HOWTO.pdf, bare_adv.tex,
%%                    bare_conf.tex, bare_jrnl.tex, bare_jrnl_compsoc.tex,
%%                    bare_jrnl_transmag.tex
%%*************************************************************************

% Note that the a4paper option is mainly intended so that authors in
% countries using A4 can easily print to A4 and see how their papers will
% look in print - the typesetting of the document will not typically be
% affected with changes in paper size (but the bottom and side margins will).
% Use the testflow package mentioned above to verify correct handling of
% both paper sizes by the user's LaTeX system.
%
% Also note that the "draftcls" or "draftclsnofoot", not "draft", option
% should be used if it is desired that the figures are to be displayed in
% draft mode.
%
%\documentclass[journal]{IEEEtran}
\documentclass[12pt,draft,onecolumn]{IEEEtran}
%
% If IEEEtran.cls has not been installed into the LaTeX system files,
% manually specify the path to it like:
% \documentclass[journal]{../sty/IEEEtran}





% Some very useful LaTeX packages include:
% (uncomment the ones you want to load)


% *** MISC UTILITY PACKAGES ***
%
%\usepackage{ifpdf}
% Heiko Oberdiek's ifpdf.sty is very useful if you need conditional
% compilation based on whether the output is pdf or dvi.
% usage:
% \ifpdf
%   % pdf code
% \else
%   % dvi code
% \fi
% The latest version of ifpdf.sty can be obtained from:
% http://www.ctan.org/tex-archive/macros/latex/contrib/oberdiek/
% Also, note that IEEEtran.cls V1.7 and later provides a builtin
% \ifCLASSINFOpdf conditional that works the same way.
% When switching from latex to pdflatex and vice-versa, the compiler may
% have to be run twice to clear warning/error messages.






% *** CITATION PACKAGES ***
%
%\usepackage{cite}
% cite.sty was written by Donald Arseneau
% V1.6 and later of IEEEtran pre-defines the format of the cite.sty package
% \cite{} output to follow that of IEEE. Loading the cite package will
% result in citation numbers being automatically sorted and properly
% "compressed/ranged". e.g., [1], [9], [2], [7], [5], [6] without using
% cite.sty will become [1], [2], [5]--[7], [9] using cite.sty. cite.sty's
% \cite will automatically add leading space, if needed. Use cite.sty's
% noadjust option (cite.sty V3.8 and later) if you want to turn this off
% such as if a citation ever needs to be enclosed in parenthesis.
% cite.sty is already installed on most LaTeX systems. Be sure and use
% version 4.0 (2003-05-27) and later if using hyperref.sty. cite.sty does
% not currently provide for hyperlinked citations.
% The latest version can be obtained at:
% http://www.ctan.org/tex-archive/macros/latex/contrib/cite/
% The documentation is contained in the cite.sty file itself.






% *** GRAPHICS RELATED PACKAGES ***
%
\ifCLASSINFOpdf
  % \usepackage[pdftex]{graphicx}
  % declare the path(s) where your graphic files are
  % \graphicspath{{../pdf/}{../jpeg/}}
  % and their extensions so you won't have to specify these with
  % every instance of \includegraphics
  % \DeclareGraphicsExtensions{.pdf,.jpeg,.png}
\else
  % or other class option (dvipsone, dvipdf, if not using dvips). graphicx
  % will default to the driver specified in the system graphics.cfg if no
  % driver is specified.
  % \usepackage[dvips]{graphicx}
  % declare the path(s) where your graphic files are
  % \graphicspath{{../eps/}}
  % and their extensions so you won't have to specify these with
  % every instance of \includegraphics
  % \DeclareGraphicsExtensions{.eps}
\fi
% graphicx was written by David Carlisle and Sebastian Rahtz. It is
% required if you want graphics, photos, etc. graphicx.sty is already
% installed on most LaTeX systems. The latest version and documentation
% can be obtained at: 
% http://www.ctan.org/tex-archive/macros/latex/required/graphics/
% Another good source of documentation is "Using Imported Graphics in
% LaTeX2e" by Keith Reckdahl which can be found at:
% http://www.ctan.org/tex-archive/info/epslatex/
%
% latex, and pdflatex in dvi mode, support graphics in encapsulated
% postscript (.eps) format. pdflatex in pdf mode supports graphics
% in .pdf, .jpeg, .png and .mps (metapost) formats. Users should ensure
% that all non-photo figures use a vector format (.eps, .pdf, .mps) and
% not a bitmapped formats (.jpeg, .png). IEEE frowns on bitmapped formats
% which can result in "jaggedy"/blurry rendering of lines and letters as
% well as large increases in file sizes.
%
% You can find documentation about the pdfTeX application at:
% http://www.tug.org/applications/pdftex





% *** MATH PACKAGES ***
%
%\usepackage[cmex10]{amsmath}
% A popular package from the American Mathematical Society that provides
% many useful and powerful commands for dealing with mathematics. If using
% it, be sure to load this package with the cmex10 option to ensure that
% only type 1 fonts will utilized at all point sizes. Without this option,
% it is possible that some math symbols, particularly those within
% footnotes, will be rendered in bitmap form which will result in a
% document that can not be IEEE Xplore compliant!
%
% Also, note that the amsmath package sets \interdisplaylinepenalty to 10000
% thus preventing page breaks from occurring within multiline equations. Use:
%\interdisplaylinepenalty=2500
% after loading amsmath to restore such page breaks as IEEEtran.cls normally
% does. amsmath.sty is already installed on most LaTeX systems. The latest
% version and documentation can be obtained at:
% http://www.ctan.org/tex-archive/macros/latex/required/amslatex/math/





% *** SPECIALIZED LIST PACKAGES ***
%
%\usepackage{algorithmic}
% algorithmic.sty was written by Peter Williams and Rogerio Brito.
% This package provides an algorithmic environment fo describing algorithms.
% You can use the algorithmic environment in-text or within a figure
% environment to provide for a floating algorithm. Do NOT use the algorithm
% floating environment provided by algorithm.sty (by the same authors) or
% algorithm2e.sty (by Christophe Fiorio) as IEEE does not use dedicated
% algorithm float types and packages that provide these will not provide
% correct IEEE style captions. The latest version and documentation of
% algorithmic.sty can be obtained at:
% http://www.ctan.org/tex-archive/macros/latex/contrib/algorithms/
% There is also a support site at:
% http://algorithms.berlios.de/index.html
% Also of interest may be the (relatively newer and more customizable)
% algorithmicx.sty package by Szasz Janos:
% http://www.ctan.org/tex-archive/macros/latex/contrib/algorithmicx/




% *** ALIGNMENT PACKAGES ***
%
%\usepackage{array}
% Frank Mittelbach's and David Carlisle's array.sty patches and improves
% the standard LaTeX2e array and tabular environments to provide better
% appearance and additional user controls. As the default LaTeX2e table
% generation code is lacking to the point of almost being broken with
% respect to the quality of the end results, all users are strongly
% advised to use an enhanced (at the very least that provided by array.sty)
% set of table tools. array.sty is already installed on most systems. The
% latest version and documentation can be obtained at:
% http://www.ctan.org/tex-archive/macros/latex/required/tools/


% IEEEtran contains the IEEEeqnarray family of commands that can be used to
% generate multiline equations as well as matrices, tables, etc., of high
% quality.




% *** SUBFIGURE PACKAGES ***
%\ifCLASSOPTIONcompsoc
%  \usepackage[caption=false,font=normalsize,labelfont=sf,textfont=sf]{subfig}
%\else
%  \usepackage[caption=false,font=footnotesize]{subfig}
%\fi
% subfig.sty, written by Steven Douglas Cochran, is the modern replacement
% for subfigure.sty, the latter of which is no longer maintained and is
% incompatible with some LaTeX packages including fixltx2e. However,
% subfig.sty requires and automatically loads Axel Sommerfeldt's caption.sty
% which will override IEEEtran.cls' handling of captions and this will result
% in non-IEEE style figure/table captions. To prevent this problem, be sure
% and invoke subfig.sty's "caption=false" package option (available since
% subfig.sty version 1.3, 2005/06/28) as this is will preserve IEEEtran.cls
% handling of captions.
% Note that the Computer Society format requires a larger sans serif font
% than the serif footnote size font used in traditional IEEE formatting
% and thus the need to invoke different subfig.sty package options depending
% on whether compsoc mode has been enabled.
%
% The latest version and documentation of subfig.sty can be obtained at:
% http://www.ctan.org/tex-archive/macros/latex/contrib/subfig/




% *** FLOAT PACKAGES ***
%
%\usepackage{fixltx2e}
% fixltx2e, the successor to the earlier fix2col.sty, was written by
% Frank Mittelbach and David Carlisle. This package corrects a few problems
% in the LaTeX2e kernel, the most notable of which is that in current
% LaTeX2e releases, the ordering of single and double column floats is not
% guaranteed to be preserved. Thus, an unpatched LaTeX2e can allow a
% single column figure to be placed prior to an earlier double column
% figure. The latest version and documentation can be found at:
% http://www.ctan.org/tex-archive/macros/latex/base/


%\usepackage{stfloats}
% stfloats.sty was written by Sigitas Tolusis. This package gives LaTeX2e
% the ability to do double column floats at the bottom of the page as well
% as the top. (e.g., "\begin{figure*}[!b]" is not normally possible in
% LaTeX2e). It also provides a command:
%\fnbelowfloat
% to enable the placement of footnotes below bottom floats (the standard
% LaTeX2e kernel puts them above bottom floats). This is an invasive package
% which rewrites many portions of the LaTeX2e float routines. It may not work
% with other packages that modify the LaTeX2e float routines. The latest
% version and documentation can be obtained at:
% http://www.ctan.org/tex-archive/macros/latex/contrib/sttools/
% Do not use the stfloats baselinefloat ability as IEEE does not allow
% \baselineskip to stretch. Authors submitting work to the IEEE should note
% that IEEE rarely uses double column equations and that authors should try
% to avoid such use. Do not be tempted to use the cuted.sty or midfloat.sty
% packages (also by Sigitas Tolusis) as IEEE does not format its papers in
% such ways.
% Do not attempt to use stfloats with fixltx2e as they are incompatible.
% Instead, use Morten Hogholm'a dblfloatfix which combines the features
% of both fixltx2e and stfloats:
%
% \usepackage{dblfloatfix}
% The latest version can be found at:
% http://www.ctan.org/tex-archive/macros/latex/contrib/dblfloatfix/




%\ifCLASSOPTIONcaptionsoff
%  \usepackage[nomarkers]{endfloat}
% \let\MYoriglatexcaption\caption
% \renewcommand{\caption}[2][\relax]{\MYoriglatexcaption[#2]{#2}}
%\fi
% endfloat.sty was written by James Darrell McCauley, Jeff Goldberg and 
% Axel Sommerfeldt. This package may be useful when used in conjunction with 
% IEEEtran.cls'  captionsoff option. Some IEEE journals/societies require that
% submissions have lists of figures/tables at the end of the paper and that
% figures/tables without any captions are placed on a page by themselves at
% the end of the document. If needed, the draftcls IEEEtran class option or
% \CLASSINPUTbaselinestretch interface can be used to increase the line
% spacing as well. Be sure and use the nomarkers option of endfloat to
% prevent endfloat from "marking" where the figures would have been placed
% in the text. The two hack lines of code above are a slight modification of
% that suggested by in the endfloat docs (section 8.4.1) to ensure that
% the full captions always appear in the list of figures/tables - even if
% the user used the short optional argument of \caption[]{}.
% IEEE papers do not typically make use of \caption[]'s optional argument,
% so this should not be an issue. A similar trick can be used to disable
% captions of packages such as subfig.sty that lack options to turn off
% the subcaptions:
% For subfig.sty:
% \let\MYorigsubfloat\subfloat
% \renewcommand{\subfloat}[2][\relax]{\MYorigsubfloat[]{#2}}
% However, the above trick will not work if both optional arguments of
% the \subfloat command are used. Furthermore, there needs to be a
% description of each subfigure *somewhere* and endfloat does not add
% subfigure captions to its list of figures. Thus, the best approach is to
% avoid the use of subfigure captions (many IEEE journals avoid them anyway)
% and instead reference/explain all the subfigures within the main caption.
% The latest version of endfloat.sty and its documentation can obtained at:
% http://www.ctan.org/tex-archive/macros/latex/contrib/endfloat/
%
% The IEEEtran \ifCLASSOPTIONcaptionsoff conditional can also be used
% later in the document, say, to conditionally put the References on a 
% page by themselves.




% *** PDF, URL AND HYPERLINK PACKAGES ***
%
%\usepackage{url}
% url.sty was written by Donald Arseneau. It provides better support for
% handling and breaking URLs. url.sty is already installed on most LaTeX
% systems. The latest version and documentation can be obtained at:
% http://www.ctan.org/tex-archive/macros/latex/contrib/url/
% Basically, \url{my_url_here}.




% *** Do not adjust lengths that control margins, column widths, etc. ***
% *** Do not use packages that alter fonts (such as pslatex).         ***
% There should be no need to do such things with IEEEtran.cls V1.6 and later.
% (Unless specifically asked to do so by the journal or conference you plan
% to submit to, of course. )


% correct bad hyphenation here
%\hyphenation{op-tical net-works semi-conduc-tor}


\begin{document}
%
% paper title
% can use linebreaks \\ within to get better formatting as desired
% Do not put math or special symbols in the title.
\title{Variations in Tracking in Relation to Geographic Location}
%
%
% author names and IEEE memberships
% note positions of commas and nonbreaking spaces ( ~ ) LaTeX will not break
% a structure at a ~ so this keeps an author's name from being broken across
% two lines.
% use \thanks{} to gain access to the first footnote area
% a separate \thanks must be used for each paragraph as LaTeX2e's \thanks
% was not built to handle multiple paragraphs
%

\author{ Nathaniel Fruchter,
        Hsin Miao,
        Scott Stevenson
        }% <-this % stops a space
% note the % following the last \IEEEmembership and also \thanks - 
% these prevent an unwanted space from occurring between the last author name
% and the end of the author line. i.e., if you had this:
% 
% \author{....lastname \thanks{...} \thanks{...} }
%                     ^------------^------------^----Do not want these spaces!
%
% a space would be appended to the last name and could cause every name on that
% line to be shifted left slightly. This is one of those "LaTeX things". For
% instance, "\textbf{A} \textbf{B}" will typeset as "A B" not "AB". To get
% "AB" then you have to do: "\textbf{A}\textbf{B}"
% \thanks is no different in this regard, so shield the last } of each \thanks
% that ends a line with a % and do not let a space in before the next \thanks.
% Spaces after \IEEEmembership other than the last one are OK (and needed) as
% you are supposed to have spaces between the names. For what it is worth,
% this is a minor point as most people would not even notice if the said evil
% space somehow managed to creep in.



% The paper headers
%\markboth{Journal of \LaTeX\ Class Files,~Vol.~11, No.~4, December~2012}%
%{Shell \MakeLowercase{\textit{et al.}}: Bare Demo of IEEEtran.cls for Journals}
% The only time the second header will appear is for the odd numbered pages
% after the title page when using the twoside option.
% 
% *** Note that you probably will NOT want to include the author's ***
% *** name in the headers of peer review papers.                   ***
% You can use \ifCLASSOPTIONpeerreview for conditional compilation here if
% you desire.




% If you want to put a publisher's ID mark on the page you can do it like
% this:
%\IEEEpubid{0000--0000/00\$00.00~\copyright~2012 IEEE}
% Remember, if you use this you must call \IEEEpubidadjcol in the second
% column for its text to clear the IEEEpubid mark.



% use for special paper notices
\IEEEspecialpapernotice{\{nhf,hsinm,sbsteven\}@andrew.cmu.edu}




% make the title area
\maketitle

% As a general rule, do not put math, special symbols or citations
% in the abstract or keywords.
%\begin{abstract}
%The abstract goes here.
%\end{abstract}

% Note that keywords are not normally used for peerreview papers.
%\begin{IEEEkeywords}
%IEEEtran, journal, \LaTeX, paper, template.
%\end{IEEEkeywords}






% For peer review papers, you can put extra information on the cover
% page as needed:
% \ifCLASSOPTIONpeerreview
% \begin{center} \bfseries EDICS Category: 3-BBND \end{center}
% \fi
%
% For peerreview papers, this IEEEtran command inserts a page break and
% creates the second title. It will be ignored for other modes.
\IEEEpeerreviewmaketitle



\section{Introduction}
% The very first letter is a 2 line initial drop letter followed
% by the rest of the first word in caps.
% 
% form to use if the first word consists of a single letter:
% \IEEEPARstart{A}{demo} file is ....
% 
% form to use if you need the single drop letter followed by
% normal text (unknown if ever used by IEEE):
% \IEEEPARstart{A}{}demo file is ....
% 
% Some journals put the first two words in caps:
% \IEEEPARstart{T}{his demo} file is ....
% 
% Here we have the typical use of a "T" for an initial drop letter
% and "HIS" in caps to complete the first word.
\IEEEPARstart{P}{rivacy} 
Privacy laws have been enacted worldwide with the purpose of protecting internet users' private information. Privacy laws can be divided into four main models \cite{IAPPbook} that differ in scope, enforcement, and adjudication: the comprehensive model, the sectoral model, the co-regulatory model, and mixed/no-policy model. These models impact how countries handle privacy both legally and culturally, specifically in the realms of online tracking and privacy legislation. Besides, we discovered that websites utilize a diverse plethora of trackers for various purposes. However, there is currently a lack of information as to how trackers differ between countries that employ these different models. The purpose of this project is to find the relationship between the number of trackers and different countries. 

In this project, we compared the amount of trackers on websites that operate in various countries with different privacy models. We have chosen Germany to represent the comprehensive model, the United States and Japan to represent the sectoral model, and Australia to represent the co-regulatory model. The sites that we are interested in are Alexa Top 500 sites \cite{Alexa} that have domains in multiple countries. We utilized Amazon Web Services to visit and crawl the data from the websites by servers in those countries. 

We locate and identify these trackers using 3rd party HTTP requests and cookies. In addition, we identify ads from the websites by using a list provided by AdBlock browser extension \cite{adblock}. Automation of the process are handled using the OpenWPM \cite{openwpm} tool which allows for synchronization across browsers and virtual machines ensuring that requests will occur at the same time. 

In the following sections, we will first review some related literatures. Detail description of our method and experimental results are stated in Section III and IV. Discussion and future works are described in Section V and VI.

% You must have at least 2 lines in the paragraph with the drop letter
% (should never be an issue)


%\subsection{Subsection Heading Here}
%Subsection text here.

% needed in second column of first page if using \IEEEpubid
%\IEEEpubidadjcol

%\subsubsection{Subsubsection Heading Here}
%Subsubsection text here.


% An example of a floating figure using the graphicx package.
% Note that \label must occur AFTER (or within) \caption.
% For figures, \caption should occur after the \includegraphics.
% Note that IEEEtran v1.7 and later has special internal code that
% is designed to preserve the operation of \label within \caption
% even when the captionsoff option is in effect. However, because
% of issues like this, it may be the safest practice to put all your
% \label just after \caption rather than within \caption{}.
%
% Reminder: the "draftcls" or "draftclsnofoot", not "draft", class
% option should be used if it is desired that the figures are to be
% displayed while in draft mode.
%
%\begin{figure}[!t]
%\centering
%\includegraphics[width=2.5in]{myfigure}
% where an .eps filename suffix will be assumed under latex, 
% and a .pdf suffix will be assumed for pdflatex; or what has been declared
% via \DeclareGraphicsExtensions.
%\caption{Simulation Results.}
%\label{fig_sim}
%\end{figure}

% Note that IEEE typically puts floats only at the top, even when this
% results in a large percentage of a column being occupied by floats.


% An example of a double column floating figure using two subfigures.
% (The subfig.sty package must be loaded for this to work.)
% The subfigure \label commands are set within each subfloat command,
% and the \label for the overall figure must come after \caption.
% \hfil is used as a separator to get equal spacing.
% Watch out that the combined width of all the subfigures on a 
% line do not exceed the text width or a line break will occur.
%
%\begin{figure*}[!t]
%\centering
%\subfloat[Case I]{\includegraphics[width=2.5in]{box}%
%\label{fig_first_case}}
%\hfil
%\subfloat[Case II]{\includegraphics[width=2.5in]{box}%
%\label{fig_second_case}}
%\caption{Simulation results.}
%\label{fig_sim}
%\end{figure*}
%
% Note that often IEEE papers with subfigures do not employ subfigure
% captions (using the optional argument to \subfloat[]), but instead will
% reference/describe all of them (a), (b), etc., within the main caption.


% An example of a floating table. Note that, for IEEE style tables, the 
% \caption command should come BEFORE the table. Table text will default to
% \footnotesize as IEEE normally uses this smaller font for tables.
% The \label must come after \caption as always.
%
%\begin{table}[!t]
%% increase table row spacing, adjust to taste
%\renewcommand{\arraystretch}{1.3}
% if using array.sty, it might be a good idea to tweak the value of
% \extrarowheight as needed to properly center the text within the cells
%\caption{An Example of a Table}
%\label{table_example}
%\centering
%% Some packages, such as MDW tools, offer better commands for making tables
%% than the plain LaTeX2e tabular which is used here.
%\begin{tabular}{|c||c|}
%\hline
%One & Two\\
%\hline
%Three & Four\\
%\hline
%\end{tabular}
%\end{table}


% Note that IEEE does not put floats in the very first column - or typically
% anywhere on the first page for that matter. Also, in-text middle ("here")
% positioning is not used. Most IEEE journals use top floats exclusively.
% Note that, LaTeX2e, unlike IEEE journals, places footnotes above bottom
% floats. This can be corrected via the \fnbelowfloat command of the
% stfloats package.


\section{Related Work}
Privacy in the news seems inescapable; a general concern regarding the intrusiveness and pervasiveness of online tracking, advertising, monitoring has caught the public attention. For example, concerns over the activities of social networking sites and advertisers such as Facebook \cite{wsj_fb} bring up issues of anonymity and tracking in daily life. Similarly, the level of privacy protection put into place by industry giants such as Google has come under scrutiny \cite{Google_EU_marketingland} as jurisdictions with more comprehensive privacy regulations have called the effectiveness of their protections into question.

These worries also demonstrate the large amount of change that the Internet has undergone in a relatively short amount of time. As Mayer and Mitchell note \cite{Mayer_Mitchell}, individual instances of web content have evolved from a single-origin affair into a conglomeration of  ``myriad unrelated  `third-party' websites," each facilitating anything from advertising to social media. Furthermore, this explosion of third parties has existed an environment with little to no regulation until very recently \cite{Mayer_Mitchell}, with advances only occurring in the comprehensive regulatory environment provided by the European Union.  

\subsection{Privacy-related Web Measurement}
This tangled web of privacy, regulation, and jurisdiction raises many concerns. Coupled with the increased salience of online privacy concerns, this has led to an explosion of privacy-related web measurement studies in recent years. For example, Engelhardt et al. \cite{openwpm_article} have identified 32 studies that they categorize as ``web privacy measurement studies." This category of study has great breadth, ranging from technical analyses of information leaked by web scripting languages \cite{jang} to empirical analyses of search engine personalization \cite{hannak}.  In this vein, numerous comparison-style studies have also been run, touching on diverse subjects such as discrimination in online advertising \cite{sweeney} and the effectiveness of online privacy tools \cite{balebako}.

While the above studies make valuable contributions by taking on tasks like revealing the sources of potential privacy harms, detailing the effects of these third party entities, and taking a user-centric to studying and enhancing privacy, they generally do not explore the impact of industry and country-level policy on the overall incidence of these third parties. Connolly \cite{connolly} comes the closest, performing an evaluation of various websites' compliance with the European Union's ``Safe Harbor" privacy policy. Finding an astoundingly small subset of companies in compliance with Safe Harbor directives, Connolly discusses the ``significant" privacy risk to consumers resulting from noncompliance.  Issues like these raise the necessity for a more comprehensive measurement of jurisdictional differences in tracking and advertising activity (see Background and Motivation).

\subsection{Web Measurement Methodology}
Web measurement studies are considered challenging for two reasons: causality and automation \cite{openwpm_article}. These difficulties make researchers design experiments that lead to inconsistency and reinvention. In order to solve the problem, Englehardt et al. conducted a study that reviewed general experimental frameworks and performed methodological analyses of extant web measurement studies. With this framework in mind, the authors developed a platform, OpenWPM \cite{openwpm}, that addressed many of the issues of flexibility and scalability surrounding past web measurement studies. The current study will be performed using this platform, as it builds upon proven frameworks such as Selenium \cite{Selenium} and has been validated in several studies \cite{openwpm}\cite{openwpm_article}. We hope to build upon this framework and avoid further problems, especially those surrounding replication of effort and methodological inconsistencies.

\section{Method}
In our proposal, we discussed using Tor as a proxy to set our location. Tor allows users to specify exit nodes using 2-letter country codes and restricts exiting traffic to nodes in that country. This is done using a simple command-line flag when starting Tor. Tor utilizes the SOCKS5 (Socket Secure) protocol when it is operating as a proxy \cite{torsocks5}. The OpenWPM tool we are using to collect our results utilizes and relies upon a Python library called MITMProxy. This library allows OpenWPM to collect all requests and cookies at the proxy and store the results, including those that would normally be encrypted (hence the man-in-the-middle nature of the proxy). Our proposed architecture would have required the proxy created by OpenWPM to communicate with Tor directly using SOCKS. Unfortunately, MITMProxy has known problems communicating with upstream SOCKS proxies and we could not obtain results using this combination of technologies \cite{MITM}. Due to this, we were forced to look for an alternative method.

The best alternative we found was Amazon Web Services (AWS) EC2. AWS provides cloud-based virtual machines that can be configured in numerous ways. AWS employs a 'pay-for-what-you-use' model, so it is economically convenient for us to use. We installed OpenWPM on these machines and ran our tests from the cloud without having to rely on a proxy to set our location. AWS offers virtual machines in any of the following places: Virginia (US), Ireland (EU), Frankfurt (EU), Oregon (US), California (US), Singapore (Asia), Sydney (AUS), Sao Paolo (South America), and Tokyo (JP) \cite{amazonregion}. This covers almost all of the regions we would like to examine -- the only regions not represented are Russia and China which are currently not options when using AWS EC2.

\subsection{OpenWPM}

The next step is to collect data on a number of metrics related to tracking, including the number of cookies and HTTP requests. Engelhardt et al.'s OpenWPM platform is a purpose-built web measurement platform that logs a large amount of web session data in a standardized SQLite database format, making this study a perfect environment in which to use the platform. We utilized the most recent publicly available version of OpenWPM for the data collection portion of our study and used the platform?s API to programmatically crawl a list of the top 25 websites as defined by Alexa \cite{Alexa}. OpenWPM?s ?headless? Firefox backend was used for the crawl with both JavaScript and Flash enabled.

\subsection{Heuristic: First vs. Third Party Origins}

Two of our variables of interest are located within different SQLite tables generated with each OpenWPM crawl: cookies and http\_requests. We extracted the domains of cookies and the full URLs of HTTP requests from these two tables by using the sqlite3 library in Python. In order to further analyze first-party and third-party cookies and HTTP requests, we set a rule to determine whether the hyperlink in a record is related to the website where the record is extracted. To be more specific, if a hyperlink contains the domain name of the extracted website, it is a first-party cookie or HTTP request; otherwise, it is a third-party one. For example, if a cookie is extracted from amazon.de with the domain fls-eu.amazon.de, it is a first-party cookie because of an identical base domain. In contrast, if a cookie also extracted from amazon.de with the domain zanox.com, it is a third-party cookie because of the differing domains. By implementing these procedures, we can use statistical tools to analyze the collected data.

\subsection{Heuristic: AdBlock ``easylists"}

AdBlock Plus \cite{adblock} is a popular browser extension available for both Firefox and Chrome which allows users to filter and block elements on a webpage according to user-specified rules. As evidenced by the extension name, this capability is most often used in service of blocking advertisements, tracking code, or other content deemed annoying or objectionable. Due to its open source nature and large, international user base, AdBlock Plus provides us with a unique resource: a massive, crowd-sourced list of rules that allows us to detect the presence of advertising or tracking assets within a list of URLs and page elements. These rules are compiled in two ``easylists" \cite{easylist} provided on the AdBlock website, with one focused on ad-blocking rules and the other focused on tracker-blocking rules.

Using a similar approach to the one detailed in the last section, we extracted the full URLs of HTTP requests and responses from the OpenWPM crawl database using Python and the sqlite3 library. We then used a modified version of the adblockparser Python module to match the extracted HTTP request and response URLs against the two sets of AdBlock rules mentioned above. The number of positive ad or tracker hits were aggregated by domain, country, and rule set in order to produce summary statistics for use in further analysis.

\section{Results}
\subsection{Evaluation Metric}
The goal of our project is to discover the variation of trackers in different countries.  In our experimental design, the independent variable is country. It is a categorical variable with 4 levels if we compare the number of trackers in different countries. If we compare the trackers in different regulation models, the level of the variable is 3 because Japan and the United States both belong to sectoral model. 

There are some dependent variables for further analyses. First, we analyzed the number of third-party cookies and HTTP requests, which is closely related to online trackers. Second, because the number of third-party cookies and HTTP requests are dependent to the number of first-party ones, we looked at the proportion of third-party and first party cookies and HTTP requests to see whether the ratios are identical in different countries. Moreover, the number of first-party cookies or HTTP requests are analyzed because some sites (e.g. google) are both an analytics provider and a service provider, they may use other methods besides third-party cookies to track the information of users. 

Due to the categorical independent variable and quantitative dependent variables, we use one-way ANOVA test for the analyses. There are some assumptions for the test: the distribution of the dependent variables follows a Normal distribution, variances of the  dependent variables are identical, and the error of the dependent variables are independent. We ensure that the independent error assumption is correct because we collected the data by using servers in different countries. However, in order to obtain significant results, we should check whether other two assumptions are also valid when using ANOVA test to analyze the data.  

\subsection{HTTP requests}

\begin{table}[c]
\centering
\caption{Number of Third-party HTTP requests}
\begin{tabular}{|l|r|r|}
\hline
\begin{tabular}[c]{@{}l@{}}3rd-party\\ requests\end{tabular} & Mean  & \begin{tabular}[c]{@{}r@{}}Standard \\ Deviation\end{tabular} \\ \hline
United States                                               & 20.46 & 28.85                                                         \\ \hline
Japan                                                       & 20.96 & 21.92                                                         \\ \hline
Germany                                                     & 36.29 & 44.86                                                         \\ \hline
Australia                                                   & 26.26 & 36.90                                                         \\ \hline
\end{tabular}
\end{table}

\begin{table}[c]
\centering
\caption{Ratio of Third-party and first-party HTTP requests}
\begin{tabular}{|l|r|r|}
\hline
\begin{tabular}[c]{@{}l@{}}ratio of\\ requests\end{tabular} & Mean  & \begin{tabular}[c]{@{}r@{}}Standard \\ Deviation\end{tabular} \\ \hline
United States                                               & 1.806 & 3.331                                                         \\ \hline
Japan                                                       & 1.426 & 2.618                                                     \\ \hline
Germany                                                     & 1.654 & 3.513                                                         \\ \hline
Australia                                                   & 1.231 & 2.236                                                         \\ \hline
\end{tabular}
\end{table}

\begin{table}[c]
\centering
\caption{Number of First-party HTTP requests}
\begin{tabular}{|l|r|r|}
\hline
\begin{tabular}[c]{@{}l@{}}1st-party\\ requests\end{tabular} & Mean  & \begin{tabular}[c]{@{}r@{}}Standard \\ Deviation\end{tabular} \\ \hline
United States                                               & 55.63 & 71.77                                                         \\ \hline
Japan                                                       & 67.63 & 154.78                                                         \\ \hline
Germany                                                     & 81.25 & 81.89                                                         \\ \hline
Australia                                                   & 54.30 & 57.13                                                         \\ \hline
\end{tabular}
\end{table}


Table I, II, and III show mean and standard deviation of number of third party requests, ratio of third and first party requests, and number of first party requests in different countries. We found that the mean value of third-party and first party requests in Germany is more than other countries obviously. Although the mean values for HTTP requests in Germany are higher than other countries, the p-value for these variables are all larger than 0.05, which means the differences between different countries are not significant. Besides, we did not find the difference of the ratio of first and third party requests. 

There are two possible reasons for the result. First, the standard deviation between different countries are not identical. So the assumption of ANOVA test is violated. Second, we have only 25 samples in each country, so the results are easily affected by some outliers. We determined the existence of outliers from the large standard deviation value in these tables. In order to correctly analyze the data, more data is required.
\subsection{Cookies}

\begin{table}[c]
\centering
\caption{Number of Third-party HTTP cookies}
\begin{tabular}{|l|r|r|}
\hline
\begin{tabular}[c]{@{}l@{}}3rd-party\\ cookies\end{tabular} & Mean  & \begin{tabular}[c]{@{}r@{}}Standard \\ Deviation\end{tabular} \\ \hline
United States                                               & 22.82 & 44.165                                                         \\ \hline
Japan                                                       & 30.70 & 53.726                                                         \\ \hline
Germany                                                     & 79.26 & 128.493                                                         \\ \hline
Australia                                                   & 88.966 & 234.177                                                         \\ \hline
\end{tabular}
\end{table}

\begin{table}[c]
\centering
\caption{Ratio of Third-party and first-party HTTP cookies}
\begin{tabular}{|l|r|r|}
\hline
\begin{tabular}[c]{@{}l@{}}ratio of\\ cookies\end{tabular} & Mean  & \begin{tabular}[c]{@{}r@{}}Standard \\ Deviation\end{tabular} \\ \hline
United States                                               & 0.651 & 1.476                                                         \\ \hline
Japan                                                       & 3.003 & 7.726                                                     \\ \hline
Germany                                                     & 2.327 & 7.678                                                         \\ \hline
Australia                                                   & 1.258 & 2.330                                                         \\ \hline
\end{tabular}
\end{table}

\begin{table}[c]
\centering
\caption{Number of First-party HTTP cookies}
\begin{tabular}{|l|r|r|}
\hline
\begin{tabular}[c]{@{}l@{}}1st-party\\ requests\end{tabular} & Mean  & \begin{tabular}[c]{@{}r@{}}Standard \\ Deviation\end{tabular} \\ \hline
United States                                               & 106.14 & 157.091                                                         \\ \hline
Japan                                                       & 135.00 & 434.499                                                        \\ \hline
Germany                                                     & 209.78 & 378.571                                                         \\ \hline
Australia                                                   & 85.35 & 79.622                                                         \\ \hline
\end{tabular}
\end{table}



Table IV, V, and VI show mean and standard deviation of number of third party cookies, ratio of third and first party cookies, and number of first party cookies in different countries. We found that the number of third party cookies in Germany and Australia are more than the United States and Japan. However, the standard deviations are still large, so the difference is not significant. 

\subsection{AdBlock rules}

Using our current sample of the top 25 websites from each country, a significant difference was found in the number of hits, both by country and by privacy regulation model. Aggregating across requests and responses, we found a significant difference ($p<0.03$) between countries, with German sites reporting a slightly higher number of tracking/advertisement hits compared to other countries. 

This same trend remained evident after re-running the analysis using a series of dummy variables to represent the presence of different regulatory models. Germany, our representative of the comprehensive model, had a significantly higher ($M=7.89$) number of hits per site, as compared to our mixed ($M=5.16$) or sectoral ($M=5.07$) model countries.

While these results may seem surprising initially, the small sample of sites does not too unlikely that we are getting a skewed view of the landscape. If we are able to obtain data for a larger sample of sites (as mentioned below), significant changes may occur.

\section{Discussion}
Going into this experiment, we assume that there will be a significant difference in tracking between countries employing the different privacy models. We expect to see the most tracking in the no model and sectoral model countries, less in the co-regulatory model, and even less in the comprehensive model. We are also interested in determining if the country the website is based in, versus the country we are connecting from, plays a role in the amount of tracking. We have a slight expectation that it will, but we really don't know going into the experiment. Whether or not these assumptions have been confirmed is yet to be determined.

At the time of writing this draft, the results are incomplete. We have a script that allows us to collect the top n sites from a specified region utilizing the Alexa.com rankings. We have run tests on all regions except China and Russia for the top 25 websites in each country. After analyzing these results we did not notice anything significant (although there is some debate here) in the results. We are unsure of why this would be the case but we suspect it may have something to do with the small sample size. We will be keeping these results as a benchmark of comparison however. The top 25 websites in each region may not be representative of the region as a whole. In response, we decided to collect the top 250 websites for each region. We are currently analyzing the results for this collection of websites but it is taking a bit longer as the set is much larger.

In addition to the above, we are also going to compare the results for the top 250 websites from each region with the top 500 websites globally. It will be interesting to see how the results compare for these two data sets.

The results of HTTP requests and cookies show that although there are some differences of the number of third party requests and cookies between these countries, they are not significant because their p-values are more than 0.05. In order to obtain further results, we will analyze those data after collecting the data from top 250 websites of each country. Because the number of samples increases, we expect that the equal variance assumption of ANOVA test will be satisfied. If the assumption is then satisfied, we can determine whether the tracking behavior of top sites in these countries are similar. 

\section{Future work}
Although our study is fairly comprehensive in terms of what we are looking for, we are lacking in a few areas. The most prominent is that we lack a node in China and Russia, and therefore have no direct representation of the no privacy model regions. This directly affects our ability to measure tracking when connecting from these countries. We also cannot analyze China which may be an incredibly interesting case due to their ``Great Firewall" and general views on privacy. Extending our study to incorporate China will be possible soon as AWS EC2 is currently testing Beijing as a regional offering.

It may be valuable to conduct this study again in the future as well. Doing so would allow comparison of tracking throughout time and it may be fruitful to link certain privacy-related events with changes in tracking. For instance, if Do Not Track becomes a widely-accepted standard (like the US government is pushing for), how different will the tracking landscape look? Would tracking increase or decrease for people not utilizing Do Not Track?

Another extension to our study would be to look deeper into other methods of tracking. Third-party cookies, third-party HTTP requests, and ads don't tell the whole story. For example, even though Google has very few third-party cookies or requests, they are probably tracking users more than other websites that have many third-party cookies or requests. In a similar vein, many major service providers like Google are also their own analytics providers. We do not account for this possibility in our study, but developing methods for doing so may reveal a more complete picture.


% if have a single appendix:
%\appendix[Proof of the Zonklar Equations]
% or
%\appendix  % for no appendix heading
% do not use \section anymore after \appendix, only \section*
% is possibly needed

% use appendices with more than one appendix
% then use \section to start each appendix
% you must declare a \section before using any
% \subsection or using \label (\appendices by itself
% starts a section numbered zero.)
%


%\appendices
%\section{Proof of the First Zonklar Equation}
%Appendix one text goes here.

% you can choose not to have a title for an appendix
% if you want by leaving the argument blank
%\section{}
%Appendix two text goes here.


% use section* for acknowledgement
%\section*{Acknowledgment}
%The authors would like to thank...


% Can use something like this to put references on a page
% by themselves when using endfloat and the captionsoff option.
\ifCLASSOPTIONcaptionsoff
  \newpage
\fi



% trigger a \newpage just before the given reference
% number - used to balance the columns on the last page
% adjust value as needed - may need to be readjusted if
% the document is modified later
%\IEEEtriggeratref{8}
% The "triggered" command can be changed if desired:
%\IEEEtriggercmd{\enlargethispage{-5in}}

% references section

% can use a bibliography generated by BibTeX as a .bbl file
% BibTeX documentation can be easily obtained at:
% http://www.ctan.org/tex-archive/biblio/bibtex/contrib/doc/
% The IEEEtran BibTeX style support page is at:
% http://www.michaelshell.org/tex/ieeetran/bibtex/
%\bibliographystyle{IEEEtran}
% argument is your BibTeX string definitions and bibliography database(s)
%\bibliography{IEEEabrv,../bib/paper}
%
% <OR> manually copy in the resultant .bbl file
% set second argument of \begin to the number of references
% (used to reserve space for the reference number labels box)

\bibliographystyle{ieeetr}
\bibliography{./web} 

% biography section
% 
% If you have an EPS/PDF photo (graphicx package needed) extra braces are
% needed around the contents of the optional argument to biography to prevent
% the LaTeX parser from getting confused when it sees the complicated
% \includegraphics command within an optional argument. (You could create
% your own custom macro containing the \includegraphics command to make things
% simpler here.)
%\begin{IEEEbiography}[{\includegraphics[width=1in,height=1.25in,clip,keepaspectratio]{mshell}}]{Michael Shell}
% or if you just want to reserve a space for a photo:


% insert where needed to balance the two columns on the last page with
% biographies
%\newpage


% You can push biographies down or up by placing
% a \vfill before or after them. The appropriate
% use of \vfill depends on what kind of text is
% on the last page and whether or not the columns
% are being equalized.

%\vfill

% Can be used to pull up biographies so that the bottom of the last one
% is flush with the other column.
%\enlargethispage{-5in}



% that's all folks
\end{document}


